\documentclass[a4paper]{article}
    \usepackage[utf8]{inputenc}
    \usepackage{amsmath}
    \title{Analysis of the Increase and Decrease Algorithms for Congestion Avoidance in Computer Networks}
    \author{Original authors: Dah-Ming CHIU and Raj JAIN\\Summarized by Wenjie XU}
    \begin{document}
    \maketitle
    \section{Justification}
    \begin{itemize}
      \item Throughput and response time are the two main factors affecting the quality of service in computer networks. Both of them are a function of load. There is a point called \textsl{cliff} after which the throughput drops and the response time increases drastically. There is also a point called \textsl{knee} after which the increase in the throughput is small but a significant increase in the response time results.    
      \item Traditional congestion control schemes try to keep the network operating in the zone to the left of the \textsl{cliff}. However, this paper considers a class of congestion avoidance schemes called increase/decrease algorithms aiming to keep the network operating around \textsl{knee}.   
    \end{itemize}
    
    \section{Assumptions}
    \begin{itemize} 
    \item All the users sharing the same bottleneck will receive the same feedback.
    \item The feedback and control loop for all users are synchronous.
    \end{itemize} 
    
    \section{System Model}
    The paper considers the control model as shown in 
    \[
        y(t)=\left\{
                    \begin{array}{ll}
                      1\Longrightarrow \textrm{Increase load}\\
                      0\Longrightarrow \textrm{Decrease load}.
                    \end{array}
                  \right.
      \]
    
    \section{Main Points}
    \begin{itemize}
      \item 
      \item 
    \end{itemize}
    
    \section{Results}
    \begin{itemize}
      \item
    \end{itemize}
    
    
    \section{Future Work}
    In order to satisfy the requirements of distributed convergence to efficiency and fairness without truncation, the linear decrease policy should be multiplicative, and the linear increase policy should always 
have an additive component, and optionally it may have a multiplicative component with the coefficient no less than one.  
    
    
    \end{document}
    